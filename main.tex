%%%%%%%%%%%%%%%%%%%%%%%%%%%%%%%%%%%%%%%%%%%%%%%%%%%%%%%%%%%%%%%
%
% F1000Research is an open access publishing platform, with rapid publication times and an open and transparent peer review process. This template is for Software Tool articles, which should include the rationale for the development of the tool and details of the code used for its construction. The article should provide examples of suitable input data sets and include an example of the output that can be expected from the tool and how this output should be interpreted.
%
%%%%%%%%%%%%%%%%%%%%%%%%%%%%%%%%%%%%%%%%%%%%%%%%%%%%%%%%%%%%%%%
%
% For more detailed article preparation guidelines, please see:
% https://f1000research.com/for-authors/article-guidelines/software-tool-articles
%
% For more information on the F1000Research publishing model please see:
% https://f1000research.com/about

\documentclass[10pt,a4paper]{article}
\usepackage{f1000_styles}

%% Default: numerical citations
\usepackage[numbers]{natbib}

%% Uncomment this lines for superscript citations instead
% \usepackage[super]{natbib}

%% Uncomment these lines for author-year citations instead
% \usepackage[round]{natbib}
% \let\cite\citep

\begin{document}
\pagestyle{fancy}

\title{F1000Research Software Tool Article Template}
\titlenote{Please provide a concise and specific title that clearly reflects the content of the article.}
\author[1]{Author Name-1}
\author[2]{Author Name-2}
\affil[1]{Address of author-1}
\affil[2]{Address of author-2}

\maketitle
\thispagestyle{fancy}

Please list all authors that played a significant role in developing the software tool and/or writing the article. Please provide full affiliation information (including full institutional address, ZIP code and e-mail address) for all authors, and identify who is/are the corresponding author(s).
\\
\\
\begin{abstract}

Abstracts should be up to 300 words and provide a succinct summary of the article. Although the abstract should explain why the article might be interesting, care should be taken not to inappropriately over-emphasise the importance of the work described in the article. Citations should not be used in the abstract, and abbreviations, if needed, should be spelled out in full.

\end{abstract}

\section*{\color{f1ROrange}Keywords}

Please list up to eight relevant keywords that describe the subject of their article. These will improve the visibility of your article.


\clearpage
\pagestyle{fancy}
\section*{Introduction}

The introduction provides context as to why the software tool was developed and what need it addresses.  It is good scholarly practice to mention previously developed tools that address similar needs, and why the current tool is needed. 

\section*{Methods}
\subsection*{Implementation}
For software tool papers, this section should address how the tool works and any relevant technical details required for implementation of the tool by other developers.  

\subsection*{Operation}
This part of the methods should include the minimal system requirements needed to run the software and an overview of the workflow for the tool for users of the tool.


\section*{Results} % Optional - only if novel data or analyses are included
This section is only required if the paper includes novel data or analyses, and should be written as a traditional results section.

\section*{Use Cases} % Optional - only if no new datasets are included
This section is required if the paper does not include novel data or analyses. 
Examples of input and output files should be provided with some explanatory context.  Any novel or complex variable parameters should also be explained in sufficient detail to allow users to understand and use the tool's functionality. \textbf{Data input for use cases must be provided}

\section*{Tables}
Use \textbackslash table and \textbackslash tabledata for basic tables. See \autoref{exampletable}, for example.
\begin{table}
    \hrule height 0.05cm  \vspace{0.1cm}
	\caption{\label{exampletable}An example of a simple table with caption.}
	\centering
	\begin{tabledata}{$l^l^r} 
		\header First name & Last Name & Grade \\ 
		\row John & Doe & 7.5 \\ 
		\row Richard & Miles & 2 
	\end{tabledata}
\end{table}

\section*{Figures}
You can upload a figure (JPEG, PNG or PDF) using the files menu. To include it in your document, use  \textbackslash includegraphics (see the example in the source code below). All figures should be discussed in the article text.

Please give figures appropriate filenames eg: figure1.pdf, figure2.png.

Figure legends should briefly describe the key messages of the figure such that the figure can stand alone from the main text, and avoid lengthy descriptions of the methods. Each legend should have a concise title of no more than 15 words. Please ensure all abbreviations used in your figures and legends are defined to allow them to stand independently from the main body of the text.

If reusing a figure or table from a previous publication, the authors are responsible for obtaining permission from the copyright holder and for the payment of any fees (if applicable). Please include a note in the legend to state that: ‘This figure/table has been reproduced with permission from \textit{[include original publication citation]}’.

\begin{figure}
	\centering
	\includegraphics[width=0.8\textwidth]{F1000header.png}
	\caption{\label{fig:your-figure}Your figure legend goes here; it should be succinct, while still explaining all symbols and abbreviations. }
\end{figure}

\section*{Discussion} % Optional - only if novel data or analyses are included
This section is only required if the paper includes novel data or analyses, and should be written in the same style as a traditional discussion section.
Please include a brief discussion of allowances made (if any) for controlling bias or unwanted sources of variability, and the limitations of any novel datasets.


\section*{Conclusions} % Optional - only if novel data or analyses are included
This section is only required if the paper includes novel data or analyses, and should be written as a traditional conclusion.




\section*{Data availability} % Required
Use this section to provide the raw data that support their findings. Readers should be able to view the raw data, replicate the study, and re-analyse and/or reuse the data (with appropriate attribution). Please take a look at the F1000Research guidelines on \href{https://www.f1000research.com/for-authors/data-guidelines}{data preparation}.
Raw data should be uploaded to an approved repository before submission, a list of which can be found on the \href{https://f1000research.com/for-authors/data-guidelines#hosting}{data guidelines page}.

This section should be completed in the following format:

\subsection*{Source data}

If the data has been published previously, details of the dataset and where it can be accessed should be provided here.

\subsection*{Underlying data}

This section should detail all novel data collected and used as part of your article. Details of the repository where your data are hosted, a description of the data files and the license under which they are held should be included. See the F1000Research Data Guidelines for more information. The following formatting should be used:
\begin{quote}
Repository: Manually annotated miRNA-disease and miRNA-gene interaction corpora.\\
https://doi.org/10.5256/repository.4591.d34639.
\\
\\
This project contains the following underlying data:
\begin{itemize}
	\item Data file 1. (Description of data.)
	\item Data file 2. (Description of data.)
\end{itemize}

Data are available under the terms of the Creative Commons Zero "No rights reserved" data waiver (CC0 1.0 Public domain dedication).
\end{quote}
\subsection*{Extended data}

Additional materials that support the key claims in the paper but are not absolutely required to follow the study design and analysis of the results (e.g. questionnaires, supporting images or tables) should be included as extended data. Details of the repository where these materials are hosted, a description of the extended data files and the license under which they are held should be included. See the F1000Research Data Guidelines for more information.

\section*{Software availability}
Source code for new software must be made openly and permanently available in a structured repository such as Zenodo (see ‘Making Your Code Citable’ for more information), and assigned an open license; we strongly encourage the use of an OSS approved license, but will accept other open licenses including Creative Commons. The Software availability section must include the following information:

\begin{itemize}
	\item Software available from: URL to own website where software can be downloaded from (if available)
	\item Source code available from: URL to versioning control system (e.g. GitHub)
	\item Archived source code at time of publication: DOI for archived source code
	\item License: OSS approved license or other open license  
\end{itemize}

\section*{Competing interests}
All financial, personal, or professional competing interests for any of the authors that could be construed to unduly influence the content of the article must be disclosed and will be displayed alongside the article. If there are no relevant competing interests to declare, please add the following: 'No competing interests were disclosed'.

\section*{Grant information}
Please state who funded the work discussed in this article, whether it is your employer, a grant funder etc. Please do not list funding that you have that is not relevant to this specific piece of research. For each funder, please state the funder’s name, the grant number where applicable, and the individual to whom the grant was assigned.
If your work was not funded by any grants, please include the line: ‘The author(s) declared that no grants were involved in supporting this work.’

\section*{Acknowledgements}
This section should acknowledge anyone who contributed to the research or the article but who does not qualify as an author based on the criteria provided earlier (e.g. someone or an organization that provided writing assistance). Please state how they contributed; authors should obtain permission to acknowledge from all those mentioned in the Acknowledgements section.

Please do not list grant funding in this section.

{\small\bibliographystyle{unsrtnat}
\bibliography{sample}}

Include a reference list in your .tex file - if using a .bib file this can be generated with \textbackslash bibliography as demonstrated above. References can be listed in any standard referencing style and should be consistent between references within a given article. In-line references should be formatted using \textbackslash cite, for example \cite{Smith:2012qr} and \cite{Smith:2013jd} 


\section*{Using LaTeX}
In order to ensure smooth and successful processing of your LaTeX manuscript, please follow these guidelines. Before submitting ensure that your PDF appears correctly on Overleaf, to avoid delays in processing. You can view and outstanding errors on the Logs and Output Files tab, just to the right of the green Recompile button.

As you prepare your LaTeX manuscript, please bear in mind the following general guidelines.  
Keep it simple:

\begin{enumerate}
    \item[~]
	\begin{enumerate}
		\item Keep your LaTeX files as simple as possible; do not use elaborate local macros or highly customized style files. Preferably, use the template provided for formatting your paper.
		\item Preferably prepare only one .tex file. 
		\item Do not use external style files or packages, except for f1000styles.sty and those packages already referenced in the main.tex template. If you need additional macros, please keep them simple and include them in the .tex document preamble.
		\item Source code should be structured so that all .sty and .bst files called by the main .tex file are in the same directory as the main .tex file.
		\item AMS math commands are recommended when inserting math equations into your manuscript.
		\item When using URLs in the text these should be incorporated as hyperlinks using the \textbackslash hyperref package and \textbackslash href\{\}\{\} function where possible.
		\item References to figures and tables within the manuscript should use \textbackslash autoref\{\}
	\end{enumerate}
\end{enumerate}

References:
\begin{itemize}
	\item Reference management systems such as F1000Workspace provide options for exporting bibliographies as BibTEX files (.bib). This template contains an example of such a file, sample.bib, which can be replaced with your own.
	\item Use only the generic \textbackslash cite\{\} command for referencing in the text (like this [1] and this [2]), not other commands built on special macros. Also, make sure that there is no space between reference keynames within the braces (i.e., \textbackslash cite\{refone,reftwo,refthree\}, not \textbackslash cite\{refone, reftwo, refthree\}).
\end{itemize}

\section*{Submitting your article}
If you are using Overleaf,  either select “Submit” then F1000Research, or click “Submit to F1000Research” in the top right-hand corner. Alternatively, generate a PDF file of your project and submit this alongside a zip file containing all project files (including the source files, style files, and PDF) using our \href{https://f1000research.com/for-authors/publish-your-research}{online submission form}. 

% See this guide for more information on BibTeX:
% http://libguides.mit.edu/content.php?pid=55482&sid=406343

% For more author guidance please see:
% https://f1000research.com/for-authors/article-guidelines/software-tool-articles

% Please note that this template results in a draft pre-submission PDF document.
% Articles will be professionally typeset when accepted for publication.

% We hope you find the F1000Research LaTex template useful, please contact us if you have any feedback.

\end{document}